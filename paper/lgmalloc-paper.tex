\documentclass[sigconf,authordraft]{acmart}

\AtBeginDocument{%
  \providecommand\BibTeX{{%
    Bib\TeX}}}

\begin{document}

\settopmatter{printacmref=false}

\title{\texttt{lgmalloc}: Predictive Memory Allocator}

\author{Luca Martijn Goddijn}
\email{luca.goddijn@gmail.com}
\affiliation{%
  \institution{Independent Researcher}
  \city{Amsterdam}
  \country{Netherlands}
}

\begin{abstract}
  Memory allocation represents a critical bottleneck in multi-threaded applications.
  Traditional allocators like \texttt{tcmalloc} suffer from central contention points
  that limit scalability, while thread-local implementations require substantial
  per-thread memory overhead.
  \texttt{lgmalloc} shifts the demand-reactive paradigm that standard allocation
  algorithms use, to a preemptive and predictive heuristic approach.
  The approach leverages the fundamental observation that call sites demonstrate
  predictable allocation patterns.
  By parsing the executable's \texttt{ELF} format before the program's entry point,
  \texttt{lgmalloc} performs static analysis of call sites to reveal allocation
  patterns and generate confidence scores for size classes. Based on the confidence
  scores, the system utilizes optimized code paths to the backend allocation API,
  reducing allocation latency. When confidence scores are low, the system falls back to
  reliable, standard allocation strategies. The thread-local design eliminates lock
  contention while substantially reducing per-thread memory overhead by optimizing
  the pre-allocated size classes each thread requires.

  \texttt{lgmalloc} is available at: \url{https://github.com/Arty3/lgmalloc}
\end{abstract}

\keywords{memory allocation, static analysis, predictive optimization, thread-local storage, heuristics, heap}

\maketitle

\section{Introduction}

\section{Background and Related Work}  

\section{System Design}

\section{Implementation}

\section{Evaluation}

\section{Discussion}

\section{Conclusion}

\bibliographystyle{ACM-Reference-Format}
\bibliography{references}

\end{document}
\endinput
